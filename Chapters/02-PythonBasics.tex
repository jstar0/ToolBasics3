\chapter{Python基础语法}
\label{cp:pythonbasics}

\section{Python极简介绍}

Python是一门\textbf{通用型高级编程语言},其设计哲学强调代码的\textit{可读性和简洁性},使用\textit{缩进}来划分代码块。\textit{支持多种编程范式},包括面向对象、命令式和函数式编程。拥有\textit{动态类型系统}和\textit{垃圾回收}功能,能够轻松实现复杂的任务。\textit{Python的标准库和第三方库生态系统非常丰富},使其成为科学计算、人工智能、Web开发等领域的热门语言选择。\textit{简单易学而功能强大的特性},使Python成为编程初学者和专业开发者的共同选择。

\section{Python基础语法}

有鉴于我们经过一学年的修习,对C/C++的用法已经有了一定的了解,因此我们在这里不再从头开始讲解Python的基础语法,而是通过建立表格进行比较,以期更好地理解Python的语法特性。\\

具体的对比请见\autoref{tab:pythonvscpp}所示。

\begin{longtable}{|p{1.7cm}|p{3.8cm}|p{3.8cm}|p{4cm}|}
    \hline
    \textbf{概念/语法} & \textbf{C/C++} & \textbf{Python} & \textbf{说明} \\
    \hline
    \textbf{变量声明} & \texttt{int x = 10;} & \texttt{x = 10} & Python不需要显式声明类型,解释器根据值类型推断。 \\
    \hline
    \textbf{注释} & \texttt{// 单行注释 \newline /* 多行注释 */} & \texttt{\# 单行注释 \newline """多行注释"""} & C/C++使用\texttt{//}和\texttt{/* */},Python用\texttt{\#}或\texttt{""" """}进行注释。 \\
    \hline
    \textbf{打印输出} & \texttt{printf("Hello, World!\textbackslash n");} & \texttt{print("Hello, World!")} & C/C++使用\texttt{printf},Python用\texttt{print}。 \\
    \hline
    \textbf{条件语句} & \texttt{if (x > 10) \{ ... \} else \{ ... \}} & \texttt{if x > 10: ... else: ...} & Python用缩进代替大括号,且不需要括号包裹条件表达式。 \\
    \hline
    \textbf{循环语句} & \texttt{for (int i = 0; i < 10; i++) \{ ... \} \newline while (x) \{ \}} & \texttt{for i in range(10): ... \newline while x:} & Python的\texttt{for}循环直接使用\texttt{range()}函数,语法更加简洁。 \\
    \hline
    \textbf{函数定义} & \texttt{int add(int a, int b) \{ return a + b; \}} & \texttt{def add(a, b): return a + b} & Python使用\texttt{def}定义函数,且无需声明返回类型。 \\
    \hline
    \textbf{数组/列表} & \texttt{int arr[5] = \{1, 2, 3, 4, 5\};} & \texttt{arr = [1, 2, 3, 4, 5]} & Python使用列表(\texttt{list}),可以动态调整大小,C/C++数组大小固定。 \\
    \hline
    \textbf{指针/引用} & \texttt{int* ptr = \&x; \newline int\& ref = x;} & 不适用 & Python不支持指针,变量为引用类型,简化了内存管理。 \\
    \hline
    \textbf{内存管理} & \texttt{malloc/free} 或 \texttt{new/delete} & 由解释器自动管理内存 & Python有垃圾回收机制,C/C++需手动管理内存。 \\
    \hline
    \textbf{结构体/类} & \texttt{struct Point \{ int x; int y; \}; \newline class} & \texttt{class Point: \newline def \_\_init\_\_(self, x, y): ...} & C使用\texttt{struct},C++用\texttt{class},Python类定义简单,使用\texttt{self}指代实例。 \\
    \hline
    \textbf{模块导入} & \texttt{\#include <stdio.h>} & \texttt{import math} & C/C++通过\texttt{\#include}导入头文件,Python使用\texttt{import}导入模块。 \\
    \hline
    \textbf{异常处理} & 不适用(C);\texttt{try/catch(C++)} & \texttt{try: ... except: ...} & C++与Python都有异常处理机制,但Python语法更简单。 \\
    \hline
    \textbf{字符串处理} & \texttt{char str[] = "Hello"; \newline std::string str = "Hello";} & \texttt{str = "Hello"} & C字符串是字符数组,C++使用\texttt{std::string},Python直接使用字符串。 \\
    \hline
    \textbf{类型转换} & \texttt{(int)x} 或 \texttt{static\_cast<int>(x)} & \texttt{int(x)} & C/C++需要强制类型转换,Python用函数进行转换。 \\
    \hline
    \textbf{文件操作} & \texttt{FILE *fp = fopen("file.txt", "r"); ... fclose(fp);} & \texttt{with open("file.txt", "r") as f: ...} & Python使用\texttt{with}管理文件,自动处理关闭操作,C需要手动关闭文件。 \\
    \hline
    \textbf{命令行参数} & \texttt{int main(int argc, char* argv[])} & \texttt{import sys; sys.argv} & C通过\texttt{argc}和\texttt{argv[]}获取参数,Python通过\texttt{sys.argv}获取命令行参数。 \\
    \hline
    \textbf{库支持} & 需要手动链接库 & 内置大量标准库,且通过\texttt{pip}可安装第三方库 & Python库管理简化,无需手动链接,且有丰富的第三方库。 \\
    \hline

    \caption{Python与C/C++基础语法对比}
    \label{tab:pythonvscpp}
\end{longtable}

\section{Python语法示例}

\subsection{最佳实践}

我们按照\textit{PEP 8风格指南},合理处理空格和缩进;建立\textit{入口点检查},即在\texttt{if \_\_name\_\_ == "\_\_main\_\_"}才执行主函数;使用\textit{函数式编程},避免全局变量;添加\textit{文档字符串},方便代码阅读和维护;在\texttt{get\_user\_input}函数中,使用\texttt{try/except}处理异常;\textit{变量命名}上,使用\texttt{snake\_case}风格。完整代码如\autoref{listing:pythonbest}所示。\\

\begin{longlisting}
    \begin{minted}{python}
#!/usr/bin/env python3
"""
Python 单文件的最佳实践示例

功能:
1. 读取用户输入的数字
2. 计算并输出数字的平方
"""

def get_user_input(prompt):
    """
    获取用户输入并将其转换为浮点数

    参数:
    prompt (str): 提示用户输入的消息

    返回:
    float: 用户输入的数字
    """
    while True:
        try:
            user_input = input(prompt)
            return float(user_input)
        except ValueError:
            print("输入无效,请输入一个有效的数字。")


def calculate_square(number):
    """
    计算给定数字的平方

    参数:
    number (float): 要计算平方的数字

    返回:
    float: 输入数字的平方
    """
    return number ** 2


def main():
    """
    主函数,执行脚本的主要功能
    """
    print("欢迎使用数字平方计算器!")
    number = get_user_input("请输入一个数字: ")
    squared_value = calculate_square(number)
    print(f"{number} 的平方是 {squared_value}")


if __name__ == "__main__":
    main()

    \end{minted}
    \caption{Python最佳实践}
    \label{listing:pythonbest}
\end{longlisting}

\subsection{多文件、进程和线程}

\subsubsection{Python的多文件结构}

我们都清楚,\textbf{在大型项目中,代码往往会被分成许多个文件}。而Python的设定就是,\textit{每个文件都可以作为模块导入},通过\texttt{import}语句实现。\\

下面给出一个实例,假使我们有 \texttt{math\_tool.py} 和 \texttt{main.py} 两个文件:

\begin{longlisting}
    \begin{minted}{python}
# math_tool.py
def add(a, b):
    return a + b

def subtract(a, b):
    return a - b

# main.py
import math_tool

def main():
    x, y = 10, 5
    print(f"{x} + {y} = {math_tool.add(x, y)}")
    print(f"{x} - {y} = {math_tool.subtract(x, y)}")

if __name__ == "__main__":
    main()
    \end{minted}
    \caption{Python多文件结构示例}
    \label{listing:pythonmulti}
\end{longlisting}

在 \texttt{main.py} 中,我们通过 \texttt{import math\_tool} 导入了 \texttt{math\_tool.py} 文件,然后就可以使用其中的函数了。

\subsubsection{Python的进程}

对于\textbf{CPU密集型任务},\textbf{使用多进程},就可以\textit{通过在操作系统中启动多个进程来实现并发执行}。\textit{每个进程拥有独立的内存空间}。Python提供了\texttt{multiprocessing}模块来支持多进程编程。

我们给出一个示例,启动\texttt{5}个进程,每个进程执行\texttt{worker}函数,独立地并行处理任务。

\begin{longlisting}
    \begin{minted}{python}
from multiprocessing import Process

def worker(number):
    """模拟一些任务"""
    print(f"Worker {number} is working")

if __name__ == "__main__":
    processes = []
    for i in range(5):
        p = Process(target=worker, args=(i,))
        processes.append(p)
        p.start()

    for p in processes:
        p.join()  # 等待所有进程完成

    print("Workers all done")
    \end{minted}
    \caption{Python多进程示例}
    \label{listing:pythonprocess}
\end{longlisting}

\subsubsection{Python的线程}

类似于进程,对于\textbf{I/O密集型任务},\textbf{使用多线程},可以\textit{通过在同一进程中启动多个线程来实现并发执行}。\textit{线程共享同一进程的内存空间}。Python提供了\texttt{threading}模块来支持多线程编程。

\begin{longlisting}
    \begin{minted}{python}
import threading
import time

def worker(number):
    """模拟一些任务"""
    print(f"Worker {number} is starting")
    time.sleep(2)
    print(f"Worker {number} is done")

if __name__ == "__main__":
    threads = []
    for i in range(5):
        t = threading.Thread(target=worker, args=(i,))
        threads.append(t)
        t.start()

    for t in threads:
        t.join()  # 等待所有线程完成

    print("Workers all done")

    \end{minted}
    \caption{Python多线程示例}
    \label{listing:pythonthread}
\end{longlisting}

\subsubsection{脚注}

有必要具体展开讨论\textbf{CPU密集型任务和I/O密集型任务}为何分别推荐使用多进程和多线程。我们说:\\

使用\textbf{进程},则\textit{每个进程都有独立的内存空间},进程间通信成本高,因此更适合CPU密集型任务。使用\textbf{线程},则\textit{线程共享内存空间},开销较低,但容易出现竞争条件,因此适合I/O密集型任务。\\

实际上,我们还可以通过\texttt{concurrent.futures}模块来更方便地实现线程池和进程池。或是仅仅建立一个数组或列表,作为一个简易的线程池或进程池,然后通过\texttt{join()}方法等待所有线程或进程完成。